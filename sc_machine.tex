\scnheader{Программная модель sc-памяти}
\scnidtf{sc-machine}
\scnidtf{Программная модель семантической памяти, реализованная на основе традиционной линейной памяти и включающая средства хранения sc-конструкций и базовые средства для обработки этих конструкций, в том числе удаленного доступа к ним посредством соответствующих сетевых протоколов}
\scnrelto{программная модель}{sc-память}
\scniselement{программная модель sc-памяти на основе линейной памяти}
\scntext{основной репозиторий исходных текстов}{https://github.com/ostis-ai/sc-machine.git}
\begin{scnrelfromlist}{компонент программной системы}
    \scnitem{Реализация sc-хранилища и средств доступа к нему}
    \begin{scnindent}
        \scntext{пояснение}{В рамках текущей \textit{Программной модели sc-памяти} под \textit{sc-хранилищем} понимается компонент программной модели, осуществляющий хранение sc-конструкций и доступ к ним через программный интерфейс. В общем случае \textit{sc-хранилище} может быть реализовано по-разному. Кроме собственно \textit{sc-хранилища} \textit{Программная модель sc-памяти} включает также \textit{Реализацию файловой памяти ostis-системы}, предназначенную для хранения содержимого \textit{внутренних файлов ostis-систем}. Стоит отметить, что при переходе с \textit{Программной модели sc-памяти} на ее аппаратную реализацию файловую память ostis-системы целесообразно будет реализовывать на основе традиционной линейной памяти (во всяком случае, на первых этапах развития \textit{семантического компьютера}).}
    \end{scnindent}
    \scnitem{Реализация базового набора платформенно-зависимых sc-агентов и их общих компонентов}
    \scnitem{Реализация подсистемы взаимодействия с внешней средой с использованием сетевых протоколов}
    \scnitem{Реализация вспомогательных инструментальных средств для работы с sc-памятью}
    \scnitem{Реализация scp-интерпретатора}
\end{scnrelfromlist}
\scntext{программная документация}{http://ostis-ai.github.io/sc-machine/}
\begin{scnrelfromlist}{используемый язык программирования}
    \scnitem{C}
    \scnitem{C++}
    \scnitem{Python}
\end{scnrelfromlist}
\scntext{примечание}{Текущий вариант \textit{Программной модели sc-памяти} предполагает возможность сохранения состояния (слепка) памяти на жесткий диск и последующей загрузки из ранее сохраненного состояния. Такая возможность необходима для перезапуска системы, в случае возможных сбоев, а также при работе с исходными текстами базы знаний, когда сборка из исходных текстов сводится к формированию слепка состояния памяти, который затем помещается в \textit{Программную модель sc-памяти}.}

\subimport{./sc-memory/}{sc_memory}
\subimport{./sc-control/}{sc_control}
\subimport{./sc-search/}{sc_search}
\subimport{./sc-kpm/}{sc_kpm}
\subimport{./sc-network/}{sc_network}
\subimport{./sc-tools/}{sc_tools}
