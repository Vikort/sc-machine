\scnheader{SCn-JSON-код}
\scnidtf{scn json}
\scnidtf{scn (json)}
\scnidtf{SCn JSON}
\scnidtf{Метаязык внешнего представления \textit{sc.n-текстов}}
\scniselement{абстрактный язык}
\scniselement{линейный язык}
\scnsubset{JSON}
\begin{scnrelfromlist}{разбиение}
    \scnitem{scn-json-элемент}
    \begin{scnindent}
        \begin{scnrelfromlist}{разбиение}
            \scnitem{scn-json-спецификация sc-элементов}
            \scnitem{scn-json-спецификация sc-элементов, принадлежащих семантической окрестности заданного sc-элемента}
            \begin{scnindent}
                \begin{scnrelfromlist}{разбиение}
                    \scnitem{scn-json-спецификация связанных sc-элементов}
                    \begin{scnindent}
                        \begin{scnreltolist}{включение}
                            \scnitem{scn-json-элемент}
                        \end{scnreltolist}
                    \end{scnindent}
                    \scnitem{scn-json-спецификация инцидентных sc-коннекторов для заданного sc-элемента}
                    \begin{scnindent}
                        \begin{scnreltolist}{включение}
                            \scnitem{scn-json-коннектор}
                        \end{scnreltolist}
                    \end{scnindent}
                    \scnitem{scn-json-модификаторы}
                    \begin{scnindent}
                        \begin{scnrelfromlist}{включение}
                            \scnitem{scn-json-модификатор}
                            \begin{scnindent}
                                \begin{scnrelfromlist}{разбиение}
                                    \scnitem{scn-json-спецификация sc-элементов}
                                    \scnitem{scn-json-спецификация sc-коннекторов для заданного scn-json-модификатор}
                                    \begin{scnindent}
                                        \begin{scnrelfromlist}{разбиение}
                                            \scnitem{scn-json-коннектор}
                                        \end{scnrelfromlist}
                                    \end{scnindent}
                                \end{scnrelfromlist}
                            \end{scnindent}
                        \end{scnrelfromlist}
                    \end{scnindent}
                \end{scnrelfromlist}
            \end{scnindent}
        \end{scnrelfromlist}
    \end{scnindent}
\end{scnrelfromlist}

\scnheader{scn-json-элемент}
\scnidtf{scn-json-node}
\begin{scnrelfromlist}{понятия, специфицирующие заданный sc-элемент}
    \scnitem{sc-адрес}
    \scnitem{класс sc-элемента\scnsupergroupsign}
    \scnitem{системный идентификатор* $\cup$ основной идентификатор*}
    \scnitem{содержимое файла ostis-системы*}
    \begin{scnindent}
        \scntext{примечание}{Первые компоненты связок этого отношения являются элементами sc-памяти, соответствующие файлам ostis-системы.}
    \end{scnindent}
    \scnitem{класс файла ostis-системы\scnsupergroupsign}
    \begin{scnindent}
        \scntext{примечание}{Первые компоненты связок этого отношения являются элементами sc-памяти, соответствующие файлам ostis-системы.}
    \end{scnindent}
    \scnitem{children}
    \scnitem{struct}
    \begin{scnindent}
        \scntext{примечание}{Данное поле доступно если объект является sc-контуром, содержащим sc-контур}
    \end{scnindent}
\end{scnrelfromlist}
\scnrelfrom{пример}{Пример описания узлов на scn-json-коде}
\begin{scnindent}
    \scneqimage[15em]{../images/kpm/node_example.png}
    \scneqimage[15em]{../images/kpm/link_example.png}
    \scneqimage[15em]{../images/kpm/struct_example.png}
\end{scnindent}

\scnheader{scn-json-спецификация sc-элементов}
\scnidtf{scn-json-base-info}
\scnidtf{Базовая информация элементов scn json}
\scntext{примечание}{Каждый объект имеет базовую информацию, описывающую sc-адрес, тип и идентификатор элемента}
\begin{scnrelfromlist}{понятия, специфицирующие заданный sc-элемент}
    \scnitem{sc-адрес}
    \scnitem{класс sc-элемента\scnsupergroupsign}
    \scnitem{системный идентификатор* $\cup$ основной идентификатор*}
\end{scnrelfromlist}
\scnrelfrom{пример}{Пример базовой информации элементов scn json на scn-json-коде}
\begin{scnindent}
    \scneqimage[15em]{../images/kpm/base_info_example.png}
    \scntext{интерпретация}{Хэш sc-адреса элемента - 145278, тип элемента - 1057, идентификатор элемента - "базовая декомпозиция*"{}}
\end{scnindent}

\scnheader{scn-json-спецификация sc-элементов, принадлежащих семантической окрестности заданного sc-элемента}
\scnidtf{scn-json-children}
\scnidtf{Список связанных с родительским объектом узлов, их дуги и модификаторы}
\scntext{примечание}{Данный список содержит объекты которые описывают тройки или пятерки при помощи 3-х списков дуг, модификаторов, связанных узловб сгруппированных по модификаторам}
\scnrelfrom{пример}{Пример списка связанных с родительским объектом узлов, их дуг и модификаторов на scn-json-коде}
\begin{scnindent}
    \scneqimage[15em]{../images/kpm/children_example.png}
\end{scnindent}

\scnheader{scn-json-спецификация инцидентных sc-коннекторов для заданного sc-элемента}
\scnidtf{scn-json-arcs}
\scnidtf{Список дуг родительского элемента}
\scnrelfrom{пример}{Пример списка дуг родительского элемента на scn-json-коде}
\begin{scnindent}
    \scneqimage[15em]{../images/kpm/arcs_example.png}
\end{scnindent}

\scnheader{scn-json-коннектор}
\scnidtf{scn-json-arc}
\scnidtf{Дуга родительского элемента}
\begin{scnrelfromlist}{понятия, специфицирующие заданный sc-элемент}
    \scnitem{sc-адрес}
    \scnitem{класс sc-элемента\scnsupergroupsign}
    \scnitem{системный идентификатор* $\cup$ основной идентификатор*}
    \scnitem{direction}
\end{scnrelfromlist}
\scnrelfrom{пример}{Пример дуги родительского элемента на scn-json-коде}
\begin{scnindent}
    \scneqimage[15em]{../images/kpm/arc_example.png}
\end{scnindent}

\scnheader{scn-json-модификаторы}
\scnidtf{scn-json-modifiers}
\scnidtf{Список модификаторов дуг родительского элемента}
\scnrelfrom{пример}{Пример списка модификаторов дуг родительского элемента на scn-json-коде}
\begin{scnindent}
    \scneqimage[15em]{../images/kpm/modifiers_example.png}
\end{scnindent}

\scnheader{scn-json-модификатор}
\scnidtf{scn-json-modifier}
\scnidtf{Модификтор дуг родительского элемента}
\begin{scnrelfromlist}{понятия, специфицирующие заданный sc-элемент}
    \scnitem{sc-адрес}
    \scnitem{класс sc-элемента\scnsupergroupsign}
    \scnitem{системный идентификатор* $\cup$ основной идентификатор*}
    \scnitem{modifierArcs}
\end{scnrelfromlist}
\scnrelfrom{пример}{Пример модификтора дуг родительского элемента на scn-json-коде}
\begin{scnindent}
    \scneqimage[15em]{../images/kpm/modifier_example.png}
\end{scnindent}

\scnheader{scn-json-modifier-arcs}
\scnidtf{Список дуг модификтора дуг родительского элемента}
\scnrelfrom{пример}{Пример списка дуг модификатора дуг родительского элемента на scn-json-коде}
\begin{scnindent}
    \scneqimage[15em]{../images/kpm/modifier_arcs_example.png}
\end{scnindent}

\scnheader{scn-json-спецификация связанных sc-элементов}
\scnidtf{scn-json-linked-nodes}
\scnidtf{Список связанных узлов с родительским элементом}
\scnrelfrom{пример}{Пример списка связанных узлов на scn-json-коде}
\begin{scnindent}
    \scneqimage[15em]{../images/kpm/linked_nodes_example.png}
\end{scnindent}

\scnheader{Реализация sc-агента трансляции конструкций sc-памяти во внешний транспортный язык SCn-JSON-код}
\begin{scnrelfromlist}{используемый язык программирования}
    \scnitem{C}
    \scnitem{C++}
\end{scnrelfromlist}
\scniselement{атомарный sc-агент}
\begin{scnrelfromset}{зависимости компонента}
    \scnitem{Реализация sc-памяти}
\end{scnrelfromset}
\scntext{адрес хранилища}
\scntext{пояснение}{Текущая реализация агента осуществляет транлсяцию из внутреннего представления знаний в формат scn (json), позволяющий отобразить полученную структуру без допольнительных преобразований, в отличии от формата scs (json).}
\scnrelfrom{пример входной конструкции}{\scnfileimage[15em]{../images/kpm/scn_json_translator_input.png}}
\scnrelfrom{пример входной конструкции}{\scnfileimage[15em]{../images/kpm/scn_json_translator_input2.png}}
\begin{scnrelfromlist}{аргументы агента}
    \scnitem{question\_instance}
    \begin{scnindent}
        \scntext{пояснение}{действие, которое необходимо протранслировать, у которого есть ответ связанный отношением nrel\_answer}
        \scntext{отношение аргумента}{исходная sc-конструкция'}
    \end{scnindent}
    \scnitem{format}
    \begin{scnindent}
        \scntext{пояснение}{формат в которой необходимо протранслировать, в данном случае формат scn (json)}
        \scntext{отношение аргумента}{выходной формат'}
    \end{scnindent}
    \scnitem{lang}
    \begin{scnindent}
        \scntext{пояснение}{язык содержимого sc-ссылок и идентификаторов}
        \scntext{отношение аргумента}{ui\_rrel\_user\_lang}
    \end{scnindent}
\end{scnrelfromlist}
\begin{scnrelfromlist}{необязательные аргументы агента}
    \scnitem{filter\_set}
    \begin{scnindent}
        \scntext{пояснение}{множество фильтрации (множество отношений, которые не нужно включать в результаты трансляции)}
        \scntext{отношение аргумента}{ui\_rrel\_filter\_list}
    \end{scnindent}
\end{scnrelfromlist}

Для установления порядка отношений и элементов, а также для фильтрации ненужных отношений в отображении для данной ostis-системы, можно задать \textit{множество порядка SCn-JSON-кода} и \textit{множество фильтрации SCn-JSON-кода}
\scnheader{множество порядка SCn JSON}
\scnidtf{Элементы множества порядка SCn JSON}
\scntext{пояснение}{Элементы множества порядка SCn JSON - это множество элементов, с которых необходимо начинать трансляцию в формате scn (json), в заданном порядке.}
\scnrelfrom{пример}{\scnfileimage[15em]{../images/kpm/concept_scn_json_elements_order_set_example.png}}

\scnheader{множество фильтрации SCn JSON}
\scnidtf{Элементы множества SCn JSON}
\scntext{пояснение}{Элементы множества фильтрации SCn JSON - это множество элементов, которые необходимо удалить из трансляции в формате scn (json).}
\scnrelfrom{пример}{\scnfileimage[15em]{../images/kpm/concept_scn_json_elements_filter_set_example.png}}
